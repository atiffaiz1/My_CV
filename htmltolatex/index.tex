\documentclass[12pt,letterpaper,oneside]{article}
\usepackage{stix}
\usepackage{etaremune}
\usepackage[letterpaper,ignoreall,top=1.0cm,bottom=1.0cm,left=0.5cm,right=0.5cm,foot=1cm,head=1cm]{geometry}
\pagestyle{empty}
%  Formatting 

%  Page Specific 
\title{Barry E Moore II}
%  Stylesheets 

\usepackage{hyperref}
% commands generated by html2latex
\newcommand{\MYh}[1]{ \underline{#1} }
\newcommand{\MYcentered}[1]{ 
\begin{center}
#1
\end{center} }


\begin{document}
%  Section: Title 
\MYcentered{

\section*{Barry E. Moore II}

\subsubsection*{         10523 Brigham Road $\bullet$ Dunkirk, NY 14048
\\         (716) 785-4053 $\bullet$ \href{mailto:bmooreii@pitt.edu}{bmooreii@pitt.edu}
\\\href{http://barrymoo.github.io}{barrymoo.github.io}}}
\rule{\textwidth}{0.5pt}
%  Section: Education 


\subsection*{\MYh{Education}}
\begin{tabular*}{\textwidth}{l @{\extracolsep{\fill}} r}
\textbf{University at Buffalo, The State University of New York} & \textbf{Buffalo, NY} \\ 
\textit{Ph.D. - Chemistry}\  & \textit{June 2016}\ 
\end{tabular*}
\begin{itemize}
	\item Advised by Professor Jochen Autschbach
	\item GPA: 3.86/4.00
\end{itemize}
\begin{tabular*}{\textwidth}{l @{\extracolsep{\fill}} r}
\textbf{Molecular Response Properties Summer School} & \textbf{Blacksburg, VA} \\ 
\textit{5 Day Workshop}\  & \textit{Summer, 2015}\ 
\end{tabular*}
\begin{itemize}
	\item Quantum Mechanical Theories underlying Absorption, Emission, and Scattering
\end{itemize}
\begin{tabular*}{\textwidth}{l @{\extracolsep{\fill}} r}
\textbf{Sustainable Software for Chemistry and Materials: Software Summer School} & \textbf{Blacksburg, VA} \\ 
\textit{Intensive 2 Week Program}\  & \textit{2013}\ 
\end{tabular*}
\begin{itemize}
	\item Python Scripting and Unit Testing
	\item Linear Algebra Package (LAPACK) Computational Algorithms
	\item Massively Parallel and GPU Programming Techniques
\end{itemize}
\begin{tabular*}{\textwidth}{l @{\extracolsep{\fill}} r}
\textbf{The State University of New York at Fredonia} & \textbf{Fredonia, NY} \\ 
\textit{Bachelor of Science - Chemistry}\  & \textit{2011}\ 
\end{tabular*}
\begin{itemize}
	\item Magna Cum Laude Honors
	\item GPA: 3.58/4.00
\end{itemize}
%  Section: Work Experience 


\subsection*{\MYh{Work Experience}}
\begin{tabular*}{\textwidth}{l @{\extracolsep{\fill}} r}
\textbf{Center for Simulation and Modeling, University of Pittsburgh} & \textbf{Pittsburgh, PA} \\ 
\textit{Research Assistant Professor}\  & \textit{Feburary 2016 - Current}\ 
\end{tabular*}
\begin{itemize}
	\item Providing User Support for Researchers in High Performace Computing (HPC)
	\item Running Workshops to Educate Users on Various Software Technologies
	\item Collaborating with Faculty Members to Complete Projects in HPC
	\item Linux System Administration (RHEL 6)
	\item Queueing and Account Management (SLURM, LDAP)
\end{itemize}
\clearpage
\begin{tabular*}{\textwidth}{l @{\extracolsep{\fill}} r}
\textbf{University at Buffalo, State University of New York} & \textbf{Buffalo, NY} \\ 
\textit{Research Assistant}\  & \textit{December 2013 - January 2016}\ 
\end{tabular*}
\begin{itemize}
	\item Ability to Work on Multiple Projects
	\item Extensive Time Management Skills     
	\item Highly Optimized Work Flow Using Linux
	\item System Administration (Hardware and Software)
\end{itemize}
\begin{tabular*}{\textwidth}{l @{\extracolsep{\fill}} r}
\textbf{University at Buffalo, State University of New York} & \textbf{Buffalo, NY} \\ 
\textit{Teaching Assistant}\  & \textit{August 2011 - December 2013}\ 
\end{tabular*}
\begin{itemize}
	\item General Chemistry for Engineers I - Fall 2011
	\item General Chemistry II - Spring 2012, Spring 2013
	\item Physical Chemistry Lab - Fall 2013
\end{itemize}
\begin{tabular*}{\textwidth}{l @{\extracolsep{\fill}} r}
\textbf{State University of New York at Fredonia} & \textbf{Fredonia, NY} \\ 
\textit{Analytical Chemist}\  & \textit{June 2009 - June 2011}\ 
\end{tabular*}
\begin{itemize}
	\item Great Lakes Fish Monitoring Project
	\item Performed column purifications for mass spectroscopy analysis of polychlorinated diphenyl ethers
	\item Development of standard operating protocols
\end{itemize}
%  Section: Conferences & Presentations 


\subsection*{\MYh{Conferences \& Presentations}}
\begin{itemize}
	\item 2013 14$^\textrm{th}$ International Conference on Chiroptical Spectroscopy (poster presented)
	\item 2012 Jewett Hall 50$^\textrm{th}$ Anniversary Celebration and SUNY Fredonia Science Alumni Conference (invited speaker)
\end{itemize}
%  Section: Honors & Awards 


\subsection*{\MYh{Honors \& Awards}}\textbf{University at Buffalo, State University of New York}
\begin{itemize}
	\item 2014-2015 Rennes M\`etropole Fellowship - Spent 4 months in France working with 2 collaborators (Boris Le Guennic and Jeanne Crassous)
	\item 2013 Acceptance to Sustainable Software for Chemistry and Materials Intensive 2 Week Software Summer School Program
	\item 2013 Outstanding Poster Award - 14$^\textrm{th}$ International Conference of Chiroptical Spectroscopy
	\item 2011 Dean's Interdisciplinary - CAMBI Fellowship for Summer Research Project
\end{itemize}\textbf{State University of New York at Fredonia}
\begin{itemize}
	\item 2011 Moos Outstanding Senior Award - Chosen by SUNY Fredonia Chemistry Department
	\item 2011 Thumm Analytical Award - Highest Overall Average in Analytical Chemistry
	\item 2010 Dingledy Scholar - Sophomore and Junior Chemistry Award Based on GPA
	\item 2009 Organic Chemistry Award - Highest Overall Average in Organic Chemistry
\end{itemize}
%  Section: Skills 


\subsection*{\MYh{Skills}}
\begin{itemize}
	\item Familiarity with many quantum chemistry codes, including: NWChem, ADF, Gaussian, QChem, and Turbomole.
	\item Experience with low level programming languages (Fortran and C/C++)
	\item Highly automated and optimized work-flow with Linux command line tools and scripting (BASH, Python)
	\item Experience with the Scientific Python libraries (SciPy, NumPy, Matplotlib, Numba)
	\item Local System Administration, including: managing hardware, installation and upkeep of Linux distributions, data backup, and computer diagnostics
	\item HPC System Administration, including: installing domain science software, account management, and queueing systems (SLURM).
	\item Document typesetting with \LaTeX, HTML/CSS, and Adobe Illustrator (generated from \href{http://barrymoo.github.io/cv}{barrymoo.github.io/cv})
	\item Administration and Deployment of WordPress sites on the Enterprise Web Infrastructure at U. Pitt.
\end{itemize}
%  Section: Research & Publications 


\subsection*{\MYh{Research \& Publications}}\textbf{Interests}
\begin{itemize}
	\item Computation of various properties with Linear Response Time-Dependent Density Functional Theory (LR-TDDFT) and Coupled-Perturbed Kohn-Sham Theory (CPKS)
	\item Computing various chiroptical spectroscopic properties, including: Optical Rotatory Dispersion (ORD), Circular Dichroism (CD), Vibrational CD, Vibrational Optical Activity, and Circularly Polarized Luminescence
	\item Application of known criterion of the exact density functional to non-empirically `tune' long-range corrected density functionals which have been shown to improve property calculations
	\item Understanding common problems with the various density functionals when calculating excitations energies in LR-TDDFT
	\item Development of quantum chemistry code in C/C++ and Fortran
\end{itemize}\textbf{Publications}
\begin{etaremune}
	\item Moore II, B.; Autschbach, J. Spin-Orbit Time-Dependent Density Functional Theory Implementation of Rotatory Strengths with Hybrid Functionals. \textbf{2015}, \textit{Manuscript in Preparation}\ 
	\item Moore II, B.; Le Guennic, B.; Autschbach, J. A `Hybrid' Time-Dependent Density Functional Theory - Complete Active Space Second Approach Perturbation Theory Approach to Computing Circularly Polarized Luminescence. \textbf{2015}, \textit{Manuscript in Preparation}.
	\item Bensalah-Ledoux, A.; Pitrat, D.; Reynaldo, T.; Srebro-Hooper, M.; Moore II, B.; Autschbach, J,; Crassous, J.; Guy, S.; Guy, L. Large-Scale Synthesis of Helicene-Like Molecules for the Design of Enantiopure Thin Films with Strong Chiroptical Activity. \textbf{2016}, \textit{22}, 3333-3346.
	\item Srebro, M.; Anger, E.; Moore II, B.; Vanthuyne, N.; Roussel, C.; R\'eau, R.; Autschbach, J.; Crassous, J. Ruthenium-Grafted Vinylhelicenes: Chiroptical Properties and Redox Switching. \textit{Chem. Eur. J.}\ \textbf{2015}, \textit{21}, 17100-17115
	\item Moore II, B.; Sun, H.; Govind, N.; Kowalski, K.; Autschbach, J. Charge-Transfer vs. Charge-Transfer-Like Excitations Revisited. \textit{J. Chem. Theory. Comput.}\ \textbf{2015}, \textit{11}, 3305-3320.
	\item Saleh, N.; Moore II, B.; Srebro, M.; Vanthuyne, N.; Toupet, L.; Williams, J. A. G.; Roussel, C.; Deol, K. K.; Muller, G.; Autschbach, J.; Crassous, J. Acid/Base-Triggered Switching of Circularly Polarized Luminescence and Electronic Circular Dichroism in Organic and Organometallic Helicenes. \textit{Chem. Eur. J.}\ \textbf{2015}, \textit{21}, 1673-1681.
	\item Moore II, B.; Charaf-Eddin, A.; Planchat, A.; Adamo, C.; Autschbach, J.; Jacquemin, D. Electronic Band Shapes Calculated with Optimally Tuned Range-Separated Hybrid Functionals. \textit{J. Chem. Theory Comput.}\ \textbf{2014}, \textit{10}, 4599-4608.
	\item Jacquemin, D.; Moore II, B.; Planchat, A.; Adamo, C.; Autschbach, J. Performance of an Optimally Tuned Range-Separated Hybrid Functional for 0-0 Electronic Excitation Energies. \textit{J. Chem. Theory Comput.}\ \textbf{2014}, \textit{10}, 1677-1685.
	\item Moore II, B.; Autschbach, J. Longest-Wavelength Electronic Excitations of Linear Cyanines: The Role of Electron Delocalization and of Approximations in Time-Dependent Density Functional Theory. \textit{J. Chem. Theory Comput.}\ \textbf{2013}, \textit{9}, 4991-5003.
	\item Moore II, B.; Srebro, M.; Autschbach, J. Analysis of Optical Activity in Terms of Bonds and Lone-Pairs: The Exceptionally Large Optical Rotation of Norbornenone. \textit{J. Chem. Theory Comput.}\ \textbf{2012}, \textit{8}, 4336-4346.
	\item Pandrala, M.; Li, F.; Wallace, L.; Steel, P. J.; Moore II, B.; Autschbach, J.; Collins, J. G.; Keene, F. R. Iridium (III) Complexes Containing 1,10-Phenanthroline and Derivatives: Synthetic, Stereochemical, and Structural Studies, and Their Antimicrobial Activity. \textit{Aust. J. Chem.}\ \textbf{2013}, \textit{66}, 1065-1073.
	\item Moore II, B.; Autschbach, J. Density Functional Study of Tetraphenylporphyrin Long-Range Exciton Coupling. \textit{ChemistryOpen}\ \textbf{2012}, \textit{1}, 184-194.
\end{etaremune}

\end{document}
